% !TEX program = xelatex
% This is my resume
% Chinese translation
% by ice1000

\documentclass{style/resume2}
\usepackage{style/linespacing_fix} % disable extra space before next section
\usepackage{cite}
\usepackage{ctex}
%\usepackage[colorlinks,linkcolor=blue]{hyperref}

%\setmainfont[]{Sarasa Gothic SC}

\begin{document}
\pagenumbering{gobble} % suppress displaying page number
\hypersetup{
    colorlinks=false,
    pdfborder={0 0 0},
}

\name{张寅森}

\basicInfo{
  \email{ice1000kotlin@foxmail.com} \textperiodcentered\ 
  \phone{(+86) 180-8192-5082} \textperiodcentered\ 
  \github[ice1000]{https://github.com/ice1000}
  % \linkedin[billryan8]{https://www.linkedin.com/in/billryan8}
}

\section{\faGraduationCap\ 教育经历}
\datedsubsection{\textbf{成都外国语学校}, 成都, 中国}{2015.9 -- 2018.6}
\datedsubsection{\textbf{宾夕法尼亚州州立大学}, 美国}{2018.8 -- 现在}

\section{\faUsers\ 工作经历}
\datedsubsection{\textbf{前海源伞}, 深圳, 中国}{2018.2 -- 2018.7}
\role{实习}{编译器前端,IDE 插件开发}
\begin{itemize}
  \item 负责 pinpoint 分析器的 IntelliJ/CLion/Eclipse 工具集成,协助开发 SonarQube 插件
  \item 编写了一个多线程的跨 Java/Kotlin 的源代码索引工具,索引 hadoop 仅需 4 分钟
  \item 修改 clang-3.6 前端使其能编译目前正在使用 clang-4.0 的区块链项目 \href{https://github.com/EOSIO/eos} {eos}
\end{itemize}

\section{\faGithubAlt\ 个人项目}
% \datedsubsection{\textbf{DevKt}}{\url{https://icela.github.io/}}
% % \role{Kotlin, C\#, Racket, Ruby}{发起者和 Kotlin/C\#/Ruby 版本的主维护者}
% 一个跨语言、易使用的游戏引擎系列。
% \begin{itemize}
%   % \item 易于使用,仅需实现生命周期方法,然后调用非常方便的 API 。
%   % \item 易于安装,基于各语言内置的 GUI 框架。
%   \item 基于生命周期方法和工具 API ,并基于各语言内置的 GUI 框架,易于安装。
%   \item 使用 GitHub 的 issue 和 milestone 功能作为任务和版本管理工具,有完善的改动记录和文档。
%   \item 提供一个基于 JavaFX 的拖拽式可视化设计器,所见即所得,可以生成各种语言的代码。
% \end{itemize}
\datedsubsection{\textbf{DevKt}}{\url{https://github.com/ice1000/dev-kt}}
跨平台轻量级代码编辑器兼 Kotlin IDE。
\begin{itemize}
  \item 内置 Java/Kotlin 的高亮、补全,其他语言可以借助插件(可以以极低的成本移植 JetBrains IDE 插件) \\ 做到同样的支持。对 Kotlin 有额外的编译运行支持。
  \item 架构灵活,编辑器上层逻辑和 UI 框架彻底解藕,易于向其他 UI 框架移植。
  \item 提供细粒度的高亮颜色和快捷键设置,设置可以热更新。
\end{itemize}

\datedsubsection{\textbf{Lice 语言}}{\url{https://github.com/lice-lang/lice}}
% \role{Kotlin, Java}{发起者和主维护者}
高度可扩展的解释型程序语言,运行在 JVM 上。
\begin{itemize}
  \item 支持 lambda 和 惰性求值 (call by need) / 正则序求值 (call by name) / 严格求值 (call by value)。
\end{itemize}

\datedsubsection{\textbf{Julia-IntelliJ}}{\url{https://github.com/ice1000/julia-intellij}}
JetBrains IDE 的 Julia 插件,支持全线 JetBrains 产品。
\begin{itemize}
  \item 支持基于语义的高亮、错误检查、快速修复、定义跳转、补全、针对 Unicode 字符的特殊输入。
\end{itemize}

% Reference Test
%\datedsubsection{\textbf{Paper Title\cite{zaharia2012resilient}}}{May. 2015}
%An xxx optimized for xxx\cite{verma2015large}
%\begin{itemize}
%  \item main contribution
%\end{itemize}

\section{\faHeartO\ 成就}
\datedline{在
  \href{https://www.codewars.com/users/ice1000} {CodeWars}
  上,以 Haskell 为主,达到
  \textbf{1 kyu},
  全站排名 \#31}{Aug. 2017}

\section{\faCogs\ 技能}
\begin{itemize}[parsep=0.5ex]
  \item \textbf{程序语言}:
    \textbf{泛语言}(不受特定语言限制),
    且尤其熟悉 Java/Kotlin/Dart/C++/C\#/Haskell/Agda (不分先后)

  % language Kotlin
  \item \textbf{Kotlin/Java}:
    \textbf{2 年开发经验},
    \textbf{4} 个项目被
    \href{https://kotlin.link/?q=ice} {Awesome Kotlin}
    收录
\end{itemize}

% \section{\faHeartO\ Honors and Awards}
% \datedline{\textit{\nth{1} Prize}, Award on xxx }{Jun. 2013}
% \datedline{Other awards}{2015}

\section{\faInfo\ 其他}
\begin{itemize}[parsep=0.5ex]
  \item 获取此简历的最新版本: \url{https://tinyurl.com/ya4urea8}
\end{itemize}

%% Reference
%\newpage
%\bibliographystyle{IEEETran}
%\bibliography{mycite}
\end{document}
